\documentclass[12pt]{article}
\usepackage[T1]{fontenc}
\usepackage[utf8]{inputenc}
\usepackage[a4paper,top=1cm,bottom=2cm,margin=2cm]{geometry}

\usepackage[magyar]{babel}

\usepackage{times}

\usepackage{amsmath}
\usepackage{amssymb}
\usepackage{graphicx}
\usepackage{multirow}
\usepackage{array}
\usepackage{hyperref}

\linespread{1.1}
\setlength\parindent{0pt}
\setlength{\parskip}{0.3cm}

\newcommand\mail[1]{\href{mailto:#1}{\texttt{#1}}}

\begin{document}
 \thispagestyle{empty}

\textsc{Személyes}

       \begin{tabular}{p{3cm}l}
	       Név & Borbély Gábor \\
           Születési idő & 1989 március 23.\\
		   Tel. &  +36203333596\\
		   e-mail & \mail{borbely@math.bme.hu}\\
		   honlap & \url{math.bme.hu/~borbely} \\
                  & \url{http://hlt.bme.hu/hu/gaebor} \\
                  & \url{https://github.com/gaebor}
       \end{tabular}
 
 \textsc{Tanulmányok}
 
 \begin{tabular}{p{3cm}l}
	       2003 -- 2007 & Lovassy László Gimnázium (matematika tagozat) \\
           2007 -- 2010& BME TTK, matematika Bsc, szakdolgozat: \\
		              & Differenciálegyenletek Tanszék, Dr. Bálint Péter témavezetésével, \\
					  & \emph{Study of Decay of Correlations in a System of Two Falling Balls} \\
		   2010 -- 2013 & BME TTK, alkalmazott matematika Msc, sztochasztika szakirány,\\
						& Differenciálegyenletek Tanszék, Dr. Gyurkovics Éva témavezetésével \\
						& \emph{Application of an Abstract Multiplier Method in Model Predictive Control}\\ & \emph{ with Guaranteed Cost} \\
		   2013 -- 2016 & BME Matematika- és Számítástudományok Doktori Iskola (abszolvált)
       \end{tabular}

 \textsc{Publikációk}

        \begin{tabular}{p{3cm}l}
	       lásd & \url{https://hlt.bme.hu/hu/gaebor} \\ 
               &  \href{https://m2.mtmt.hu/gui2/?type=authors&mode=browse&sel=10043154}{\texttt{m2.mtmt.hu/gui2{\footnotesize/?type=authors\&mode=browse\&sel=10043154}}}
       \end{tabular}  

 \textsc{Munka}

       \begin{tabular}{p{4cm}l}
	       2010 ősz -- 2015 tavasz & Matematika A1, A2, A3 gyakorlatvezetés mérnököknek\\
           2011 tavasz  & Nokia Siemens Networks hálózat modellezési projekt részvétel\\ & a Matematika Intézeten keresztül. \\
		   2011 nyár    & Dolphio Consulting Kft. gyakornoki program\\
		   2012 nyár-- 2015 január & Kutató-fejlesztő (C++) a Dolphio Technologies Kft-nél \\
           2017 -- 2020 & Tanársegéd a BME Algebra tanszéken \\
           2017 nyár & SignAll projekt \\
           2020 -- 2024 & Data Scientist a \href{https://lensa.com/}{\texttt{lensa.com}}-nál \\
	   2024 --      & \href{https://deepsign.ai/}{\texttt{DeepSign.ai}}
       \end{tabular}
 
 \textsc{Technikai ismeretek}

    \begin{tabular}{p{3cm}l}
	       programozás & bash, C, C++ (GNU és Visual C++, STL és OpenCv), Matlab, \\
                             & Wolfram Mathematica, \LaTeX, Python (neurális háló függvénykönyvtárak) \\
           eszközök & Docker, Git, MS Teams, Visual Studio, AWS \\
           nyelv & angol: középfokú nyelvvizsga, szakirodalom olvasása és cikkírás, \\
                 & német: alapszint
   \end{tabular}
   
 \textsc{Egyéb}
 
    \begin{tabular}{l}
        fal- és sziklamászás\\
        2011 ősszel Az Egyetemi Mászókör (öntevékeny kör) elnöke \\
        2015-ben Mensa tag
   \end{tabular}
\end{document}
