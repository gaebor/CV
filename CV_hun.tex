\documentclass[12pt]{article}
\usepackage[T1]{fontenc}
\usepackage[utf8]{inputenc}
\usepackage[a4paper,top=1cm,bottom=2cm,margin=2cm]{geometry}

\usepackage[magyar]{babel}

\usepackage{times}

\usepackage{amsmath}
\usepackage{amssymb}
\usepackage{graphicx}
\usepackage{multirow}
\usepackage{array}
\usepackage{hyperref}

\linespread{1.1}

\newcommand\mail[1]{\href{mailto:#1}{\texttt{#1}}}

\begin{document}
{ \ }
\\
\textsc{Személyes}
 \vspace{0.3cm}
 \\
       \begin{tabular}{p{3cm}l}
	       Név & Borbély Gábor \\
           Születési idő & 1989 március 23.\\
		   Tel. &  +36203333596\\
		   e-mail & \mail{borbely@math.bme.hu}\\
		   honlap & \url{math.bme.hu/~borbely} \\
                  & \url{http://hlt.bme.hu/hu/gaebor} \\
                  & \url{https://github.com/gaebor}
       \end{tabular}
 \vspace{0.3cm}
 \\
 \textsc{Tanulmányok}
 \vspace{0.3cm}
 \\
       \begin{tabular}{p{3cm}l}
	       2003 -- 2007 & Lovassy László Gimnázium (matematika tagozat) \\
           2007 -- 2010& BME TTK, matematika Bsc, szakdolgozat: \\
		              & Differenciálegyenletek Tanszék, Dr. Bálint Péter témavezetésével, \\
					  & \emph{Study of Decay of Correlations in a System of Two Falling Balls} \\
		   2010 -- 2013 & BME TTK, alkalmazott matematika Msc, sztochasztika szakirány,\\
						& Differenciálegyenletek Tanszék, Dr. Gyurkovics Éva témavezetésével \\
						& \emph{Application of an Abstract Multiplier Method in Model Predictive Control}\\ & \emph{ with Guaranteed Cost} \\
		   2013 -- 2016 & BME Matematika- és Számítástudományok Doktori Iskola (abszolvált)
       \end{tabular}
 \vspace{0.3cm}
 \\
 \textsc{Publikációk}
 \vspace{0.3cm}
 \\
        \begin{tabular}{p{3cm}l}
	       2011 augusztus & Kaotikus dinamikai rendszerek konferencia (\href{http://www.mpipks-dresden.mpg.de/~wchaos11/}{wchaos11}) poszter díjazott %\\ & (\url{http://www.pks.mpg.de/~wchaos11/})
		   \\
			2012 & Bálint Péter, Némedy Varga András és Borbély Gábor\\
                & \emph{Statistical Properties of The System of Two Falling Balls} \\
                & Chaos 22: (2) Paper 026104
			\\
			2013 & Borbély, D., Koppányi, Z., Borbély, G., Görög, P.\\
                & \emph{Töréskép-optimalizálás alkalmazása a geotechnikában} \\
                & Magyar építőipar 2013, 3.
            \\
            2016 január & G. Recski, A. Bolevácz, G. Borbély \\
            & \emph{Building definition graphs using monolingual dictionaries of Hungarian} \\
            & XII. Magyar Számítógépes Nyelvészeti Konferencia (MSZNY 2016)
            \\
            2016 augusztus & G. Borbély, A. Kornai, M. Makrai, D. Nemeskey \\
                & \emph{Evaluating multi-sense embeddings for semantic resolution} \\
                & \emph{monolingually and in word translation} \\
                & RepEval (ACL2016) \\
            2016& Szilágyi Brigitta, Csehi Csongor György és Borbély Gábor:\\
                & \emph{On the rectifiability condition of a second order differential equation}\\
                & \emph{in special Finsler spaces}, Studies of the University of Žilina \\
                & Mathematical Series 28: pp. 5-12. (benyújtva 2011-ben)
       \end{tabular}
 \newpage
 { \ }\\ 
  \textsc{TDK}
 \vspace{0.3cm}
 \\
        \begin{tabular}{p{3cm}l}
		2010& Dr. Bálint Péter témavezetésével TTK kari TDK 2. helyezés, \\
            & \emph{Statistical behaviour in the
System of Two Falling Balls}\\
		2011& Dr. Görög Péter témavezetésével, Borbély Dániellel és Koppányi Zoltánnal\\
            & Geotechnika és geológia szekció 1. helyezett és TÁMOPT különdíj\\
            & \emph{Töréskép optimalizálás: Elmélet, megvalósítás és alkalmazás}\\
        2013& A fenti munkával OTDK, Műszaki Tudományi Szekció 3. helyezett.
       \end{tabular}
	\vspace{0.3cm}
 \\
 \textsc{Munka}
 \vspace{0.3cm}
 \\
       \begin{tabular}{p{4cm}l}
	       2010 ősz -- 2015 tavasz & Matematika A1, A2, A3 gyakorlatvezetés mérnököknek\\
           2011 tavasz  & Nokia Siemens Networks hálózat modellezési projekt részvétel\\ & a Matematika Intézeten keresztül. \\
		   2011 nyár    & Dolphio Consulting Kft. gyakornoki program\\
		   2012 nyár-- 2015 január & Kutató-fejlesztő (C++) a Dolphio Technologies Kft-nél \\
           2017 február -- & Tanársegéd a BME Algebra tanszéken \\
           2017 nyár & SignAll projekt
       \end{tabular}
 \vspace{0.3cm}
 \\
 \textsc{Egyéb}
 \vspace{0.3cm}
 \\
        \begin{tabular}{p{3cm}l}
	       program ismeretek &  C, C++ (GNU és Visual C++, STL és OpenCv függvénykönyvtár), Matlab, \\
                             & Wolfram Mathematica, \LaTeX, Python 2 és 3 (numpy és theano függvénykönyvtár) \\
           nyelv & angol: középfokú nyelvvizsga, szakirodalom olvasása és cikkírás, \\
                 & német: alapszint\\
		   közösségi tevékenység  & 2011 ősszel Az Egyetemi Mászókör (öntevékeny kör) elnöke \\
                                  & 2015-ben Mensa tag
       \end{tabular}
\end{document}